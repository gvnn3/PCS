\maketitle
\begin{frame}
	\frametitle{Introduction}
	\begin{itemize}
	\item
	\item
	\item
	\end{itemize}
\end{frame}

\begin{frame}
	\frametitle{Simple Scripts}
	\begin{itemize}
	\item
	\item
	\item
	\end{itemize}
\end{frame}

\begin{frame}
	\frametitle{}
	\begin{itemize}
	\item
	\item
	\item
	\end{itemize}
\end{frame}

\begin{frame}
	\frametitle{New Packets}
	\begin{itemize}
	\item Packet Class
	\item Layout
	\item Fields
	\end{itemize}
\end{frame}

\begin{frame}
	\frametitle{Packet Class Usage}
	\begin{lstlisting}
>>> from pcs.packets.ipv4 import *
>>> ip = ipv4()
>>> print ip
version 4
hlen 0
tos 0
length 0
id 0
flags 0
offset 0
ttl 64
protocol 0
checksum 0
src 0.0.0.0
dst 0.0.0.0
	end{lstlisting}
\end{frame}


\begin{frame}
	\frametitle{Update a Value}
	\begin{lstlisting}
>> ip.hlen=5<<2
>>> print ip
version 4
hlen 20
tos 0
length 0
id 0
flags 0
offset 0
ttl 64
protocol 0
checksum 0
src 0.0.0.0
dst 0.0.0.0
	\end{lstlisting}
\end{frame}

\begin{frame}
	\frametitle{}
	\begin{lstlisting}

	\end{lstlisting}
\end{frame}

\begin{frame}
	\frametitle{}
	\begin{lstlisting}

	\end{lstlisting}
\end{frame}


