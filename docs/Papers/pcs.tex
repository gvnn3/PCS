\documentclass[pdftex]{article}
%\usepackage{codespelunking}
\usepackage{hyperref}
\title{Packet Construction Set: A New Approach to Network Protocol
  Construction, Testing and Interrogation}
\author{George Vernon Neville-Neil}
\usepackage{listings}

\begin{document}
\lstset{language=C, escapeinside={(*@}{@*)}, numbers=left,
  basicstyle=\tiny, showspaces=false, showtabs=false, showstringspaces=false}

\maketitle
\begin{abstract}
Network protocol design, construction and testing has always been
hampered by the lack of a natural way in which to express packets as
source code.  The majority of network protocol software is written in
low level languages and is contained within operating systems
kernels.  While such implementations make sense for highly optimized
systems they are not the best approach for developing new or
experimental protocols, nor are they required for protocol conformance
tests, and other types of tools.

The Protocol Construction Set (PCS) is a new approach to writing
network protocol code that makes building new protocols, and new
protocol tools, easier and less error prone.  PCS's two key features
are a simple and clean representation of packets and packet fields in
source code, and the promotion of packets to high level language
objects that are easier to create and manipulate within programs.  In
this paper we present our completed system, both design and
implementation, as well as several tools that have been built using
it, including the Packet Debugger, a program for working with network
packets in the same way in which programmers use source code
debuggers.

\end{abstract}

\section{Introduction}

The way in which network protocols are written has been little changed
since the first versions of the ARPANet protocol~\cite{ietf:rfc46}
were built in the early 1970s.  At that time it was necessary to write
network protocols in a low level language, often assembler, and then
later C, and to place the protocol code into the core of the operating
system \cite{mckusick:fbsd}.  In order to verify the proper
functioning of the protocol test code might be written, but more
commonly two copies of the same code were run against each other to
make sure that the systems could ``talk'' to each other without error.
Experimenting with, and writing new, protocols often required putting
the protocol code into an operating system kernel because that was the
only way in which to get ahold of the necessary low level software and
hardware facilities to affect communication.  Although a protocol
could be built and run outside the kernel, where better tools might
make the job easier, this has always required porting the in kernel
facilities into libraries that can be used by user level programs.  To
this day most people develop their protocols in the kernel, where the
tools for debugging and working with code are at their bare minimum.
Being built in low level languages, protocols have always had to do
their own low level manipulation of packet data as well as express the
logic of the protocol machinery, often leading to confusing and hard
to read code.

The Packet Construction Set (PCS) is a set of language extensions that
make writing protocol code more natural and less error prone, allowing
for faster development and easier experimentation with network
protocols.  The two key features of PCS are the ability to easily
translate a packet format into a an object in the language, and the
provision of easy access to the fields of a packet as attributes of
the generated object.

The rest of this paper is organized as follows, in
Section~\ref{sec:the-programmers-view} we present how a protocol
designer, implementer, or tester uses PCS.
Section~\ref{sec:a-concrete-example} gives a concrete example of a
system built using PCS to work with several well known protocols,
including IPv4\cite{postel:rfc791}, IPv6\cite{deering:rfc2460} and
Ethernet \cite{metcalf:ethernet}.

\section{The Programmer's View}
\label{sec:the-programmers-view}

The focus in PCS is on the programmer and it is for them that the
system was built.  To most programmers the part of building a network
protocol that is tedious is the need to manipulate low level
structures, which require proper alignment, and byte ordering in order
for the protocol to work at all.  Take, for example, an IPv4 packet,
seen in Figure\ref{fig:rfc791-ipheader}.

\begin{figure}
  \centering
\label{fig:rfc791-ipheader}
%\includegraphics[width=0.5\textwidth]{IPv4Packet}
  \caption{IPv4 Header Format}
\end{figure}

IPv4 was designed in the late 1970s and first deployed in the early
1980s, when byte addressable machines were popular, computer networks
were slower and computer memory sizes were much smaller than they were
today.  In order to pack enough information into the header to make it
easy and fast to transmit the designers attempted to not waste space,
and used fields that were as efficient as they could make them.  The
IPv4 header contains 3 fields that are less than a byte in length,
including the \verb|version|, \verb|header length|, and \verb|flags.|
Though these choices meant that little space was wasted they also
require some complex gymnastics from the programmer implementing the
protocol.  Due to the fact that the \verb|flags| field is only 3 bits
in length, the \verb|Fragment Offset| starts in the middle of a
byte.  The \verb|version| and \verb|header length|, are each sub byte
quantities, which are not naturally addressable in the language in
which most IP protocol stacks are written, C.

In order to overcome these limitations programmers have written
special macros to handle getting data into and out of the fields of
the packet, and to make sure that the bytes that are placed into the
packet are in the correct byte order for the network.  These
constraints lead to code, such as that in
Figure~\ref{fig:c-structure-for-an-ipv4-packet} from the FreeBSD
Operating System Kernel \cite{mckusick:fbsd}, for representing IP
packets.

\begin{figure}
  \centering
\begin{lstlisting}
struct ip {
#if BYTE_ORDER == LITTLE_ENDIAN
	u_int	ip_hl:4,		/* header length */
		ip_v:4;			/* version */
#endif
#if BYTE_ORDER == BIG_ENDIAN
	u_int	ip_v:4,			/* version */
		ip_hl:4;		/* header length */
#endif
	u_char	ip_tos;			/* type of service */
	u_short	ip_len;			/* total length */
	u_short	ip_id;			/* identification */
	u_short	ip_off;			/* fragment offset field */
#define	IP_RF 0x8000			/* reserved fragment flag */
#define	IP_DF 0x4000			/* dont fragment flag */
#define	IP_MF 0x2000			/* more fragments flag */
#define	IP_OFFMASK 0x1fff		/* mask for fragmenting bits */
	u_char	ip_ttl;			/* time to live */
	u_char	ip_p;			/* protocol */
	u_short	ip_sum;			/* checksum */
	struct	in_addr ip_src,ip_dst;	/* source and dest address */
} __packed __aligned(4);
\end{lstlisting}
\caption{C Structure for an IPv4 Packet}
  \label{fig:c-structure-for-an-ipv4-packet}
\end{figure}

In order to access the flags field in a packet, which is not visibly
represented in this structure, the programmer needs to mask off bites
of the offset field \verb|ip_off|, or offset, field, as can be seen in
Figure~\ref{fig:accessing-the-flags-field}.

\begin{figure}
  \centering
\begin{lstlisting}
if (ip->ip_off & (IP_MF | IP_OFFMASK)) {  
\end{lstlisting}
\caption{Accessing the flags field in an IPv4 packet}
  \label{fig:accessing-the-flags-field}
\end{figure}

While the preceding code may seem reasonable and a common idiom to an
operating system kernel programmer it is not so for most other
programmers, for many people building new protocols, or for people
writing test code.  In many cases such code is not even necessary to
the task.

\subsection{PCS and Packets}
\label{sec:pcs-and-packets}

In PCS packets are transformed into objects in the Python language
\cite{vanrossum:pyref}, and their fields appear as attributes of the
object.  Each packet has two simultaneous representations, as a set of
bytes as they would be transmitted on the wire, and as a set of
attributes of an object, as used by programs written in the Python
language.

Accessing a field returns the value contained in the bytes of the
packet, and assigning a value to a field updates the bytes of the
packet.  The bytes of the packet and the values of the associated
attributes are kept in sync at all times during program execution,
which allows the programmer to ignore how the bytes are stored and
transmitted on the wire and to concentrate on the way the protocol
uses the fields of a packet.  All fields of the packet implement
bounds checking at run time, so any attempt to write a value that is
too large into the object is raised as an exception.  

Throughout the rest of this section we shall use the IPv4 packet as
our example, for all the reasons stated in
Section~\ref{sec:the-programmers-view} relating to the packet's
inherent complexity.  The code that describes the fields of an IPv4
packet in PCS is shown in Figure~\ref{fig:ipv4-packet-in-pcs} and the
full class definition can be found in the source code.

\begin{figure}
  \centering
\begin{lstlisting}
version = pcs.Field("version", 4, default = 4)
hlen = pcs.Field("hlen", 4)
tos = pcs.Field("tos", 8)
length = pcs.Field("length", 16)
id = pcs.Field("id", 16)
flags = pcs.Field("flags", 3)
offset = pcs.Field("offset", 13)
ttl = pcs.Field("ttl", 8, default = 64)
protocol = pcs.Field("protocol", 8)
checksum = pcs.Field("checksum", 16)
src = pcs.Field("src", 32)
dst = pcs.Field("dst", 32)
\end{lstlisting}
\caption{The fields of an IPv4 packet in PCS}
  \label{fig:ipv4-packet-in-pcs}
\end{figure}

A \verb|Packet| is a class defined by the PCS module, and all packets
defined in the system inherit from this base class.  In order to
define a field in a packet to be an attribute of an object, a
\verb|Field| class is used.  The \verb|Field| class can take three
arguments in its constructor.  The first argument is a \verb|name|,
which will become the name of the attribute in the object that we use
to manipulate that particular set of bits in the packet.  How that is
exploited will be covered in the next paragraphs.  The second argument
is a \verb|width|, in bits, for the field.  The \verb|width| for a
simple field can be from 1 to 32 bits.  Other types of \verb|Fields|
are provided to encode strings and length/value encoded strings into
packet, but they are beyond the scope of the present discussion.

\begin{figure}
  \centering
  \begin{lstlisting}
pcs.Packet.__init__(self,
   [version, hlen, tos, length, id, flags, 
    offset, ttl, protocol, checksum, src, dst],
   bytes = bytes)
 \end{lstlisting}
 \caption{Initializing the packet object}
  \label{fig:initializing-the-packet-object}
\end{figure}

Defining the fields in the packet is the first step in setting up a
packet object.  Once the fields are defined we must call the base
class's constructor, in Python it's \verb|__init__()| method, in order
to finishing setting up the object.  It is in the base class's
\verb|__init__()| method that the object is given the attributes we
have defined in Figure~\ref{fig:ipv4-packet-in-pcs}.  The second
argument is a list containing the fields we just defined, \emph{in the order
in which they are to appear in the packet.}  The last argument,
\verb|bytes| is used when the caller has a set of bytes from the
network that they wish to convert into an object, and its use will be
seen in Section~\ref{sec:unpacking-layers-of-packets}.

The ordering shown in the second argument is very important, because
it is this ordering that determines the way in which the fields will
be put into the underlying bytes of the packet.  The code shown in
Figure~\ref{fig:initializing-the-packet-object} exactly follows the
IPv4 packet layout shown in Figure~\ref{fig:rfc791-ipheader}.

\begin{figure}
  \centering
\begin{lstlisting}
>>> from pcs.packets.ipv4 import ipv4
>>> ip = ipv4()
>>> print ip
IPv4
version 4
hlen 0
tos 0
length 0
id 0
flags 0
offset 0
ttl 64
protocol 0
checksum 0
src 0.0.0.0
dst 0.0.0.0

>>> print ip.bytes
^@^@^@^@^@^@^@^@^@^@^@^@^@^@^@^@^@^@^@
>>> ip.hlen=5
>>> ip.ttl=32
>>> print ip.bytes
E^@^@^@^@^@^@^@ ^@^@^@^@^@^@^@^@^@^@^@
\end{lstlisting}
\caption{Interactive manipulation of an IPv4 object}
  \label{fig:interactive-manipulation-of-an-ipv4-object}
\end{figure}

With the packet so defined we can now use it in a program or work with
it interactively in the Python command line interpreter.
Figure~\ref{fig:interactive-manipulation-of-an-ipv4-object} shows the
creation of an IPv4 object from its class, and what happens when we
read or write the attributes of the object.  The \verb|bytes| field of
every \verb|Packet| object contains the up to date set of bytes for
the packet as they would be transmitted on the wire, and is not meant
to be printed or read by humans.  In this case it is instructive
because it shows, in its first and eighth locations the changes that
have been effected by changing the values of the attributes in the
IPv4 object.

PCS Packets are all derived from Python objects and therefore support
all of the expected object behavior including comparison, and human
readable display.  The human readable version of an IPv4 packet seen
in Figure~\ref{fig:interactive-manipulation-of-an-ipv4-object} is a
basic ability that all packets in \verb|PCS| inherit from their
base class.  Sub-classes may wish to override the base class method,
in order to represent some field more clearly, but this is not
strictly necessary.

\subsection{Run Time Boundary Checking}
\label{sec:run-time-boundary-checking}

In PCS, each field knows its size and so run time boundary checks are
implemented when a programmer sets a field of a packet.  Whenever a
program attempts to store a value that is too large for a field into a
packet an exception is raised, which can be caught and handled by a
program, or, when not caught, will result in the program halting.

\begin{figure}
  \centering
\begin{lstlisting}
>>> ip.ttl = 65535
Traceback (most recent call last):
[debugging information deleted]
pcs.FieldBoundsError: 
    'Value must be between 0 and 255'
>>> 
\end{lstlisting}
\caption{Run Time Boundary Checking}
  \label{fig:run-time-boundary-checking}
\end{figure}

In Figure~\ref{fig:run-time-boundary-checking} we have attempted to
assign a 16 bit value to the time to live, \verb|ttl|, field, which is
only 8 bits wide.  Boundary checking is enabled on all fields.

\subsection{Unpacking Layers of Packets}
\label{sec:unpacking-layers-of-packets}

All network protocols support layering, the ability to put one packet
into the payload of another.  The canonical example is a TCP segment
encapsulated within an IPv4 packet which is carried within an Ethernet
frame.  In PCS each packet class knows how to decode the packets that
could likely occur in its payload.  Each packet class has a
\verb|data| attribute and a \verb|next| method.  The \verb|{data|
attribute contains an object reference to the next higher layer packet
contained in the packet object, and the \verb|next| method 
calls the appropriate constructor to fill in the \verb|data|
attribute.

\begin{figure}
  \centering
\begin{lstlisting}
def next(self, bytes):
    if self.type == ETHERTYPE_ARP:
        return arp(bytes)
    if self.type == ETHERTYPE_IP:
        return ipv4(bytes)
    if self.type == ETHERTYPE_IPV6:
        return ipv6(bytes)
    return None
  \end{lstlisting}
  \caption{The Ethernet Class's next method}
  \label{fig:ethernet-next-method}
\end{figure}

Figure~\ref{fig:ethernet-next-method} shows the code from the
\verb|Ethernet| class which unpacks the protocols that can be
supported in its payload.  At the moment the \verb|Ethernet| class
can support the Address Resolution Protocol (ARP), IPv4 and IPv6.
Note that the return value from the \verb|next| method is an object
from the appropriate class for the payload found in the Ethernet packet.

\begin{figure}
  \centering
\begin{lstlisting}
if (bytes != None):
    self.data = self.next(bytes[etherlen:len(bytes)])
  \end{lstlisting}
  \caption{Calling the next method during object construction}
  \label{fig:calling-next-method}
\end{figure}

Each packet class uses its \verb|next| method to fill in its
\verb|data| attribute when its constructor,
i.e. \verb|\_\_init\_\_| method is called with packet data.
Figure~\ref{fig:calling-next-method} shows the code in the
\verb|Ethernet| class's constructor which receives the object created
by the \verb|next| method.  Packet objects are tied together via
their \verb|data| attributes, with the highest level attribute being
a \verb|Payload| class.  The \verb|Payload| class acts as a catch
all packet class which terminates a string of packets, as seen in
Figure~\ref{fig:data-field-chain}.

\begin{figure}
  \centering
\label{fig:data-field-chain}
%\includegraphics[width=0.5\textwidth]{DataFieldChain}
  \caption{Using the data field to chain packets}
\end{figure}

All packets have a method, \verb|chain|, which returns the packet's
components as a \verb|Chain| object, which is discussed in
Section~\ref{sec:layering-packets-for-output}.

\subsection{Layering Packets for Output}
\label{sec:layering-packets-for-output}

When a programmer wishes to stack up a layer of packets for output,
for instance placing an ICMP packet on top of an IPv4 packet and
encapsulating this in an Ethernet frame, they use a class called a
\verb|Chain|.  A \verb|Chain| is based on the \verb|Array| class
and supports all of its basic methods including insertion, deletion,
and slices.

As is the case with the \verb|Packet| class, a \verb|Chain| also has
a \verb|bytes| attribute, which contains the bytes of all the packets
contained in the \verb|Chain| in the proper order for transmission.
Any update to a packet within the \verb|Chain| causes the bytes
attribute to be correctly updated.  Chains can be built up in the
program and then modified and re-used on the fly because updating the
attributes of the packets in the chain updates the chain as well,
leading to more compact and efficient code for handling complex,
layered protocols.

\section{Support for reading and writing packets}
\label{sec:support-for-reading-and-writing}

So far we have described only the way in which packet classes and
packet objects are implemented and manipulated within \verb|PCS|.
For the system to be useful to a programmer it must also support the
ability to read and write data into and out of packet objects.
\verb|PCS| supports access to many types of network services
including the well known \verb|socket| library as well as directly
to network interfaces using the \verb|pcap| packet capture
facility.  Live capture and injection are supported as well as the
reading and writing of pre-recorded dump files, such as those created
with \verb|tcpdump|.

Within \verb|PCS| a special set of classes exist, collectively
known as \verb|Connectors|.  Each connector is of a specific type for
accessing a socket, network interface or dump file.  For example the
\verb|PcapConnector| can open, read, and write a network interface or
a pre-recorded dump file, whereas a \verb|TCPConnector| gives access
a more traditional TCP socket.  Each connector class implements their
own versions of the \verb|read|, \verb|write|, \verb|send|,
\verb|recv|, \verb|sendmsg|, and \verb|recvfrom| and
\verb|readpkt| methods.  In all the connector classes it is the
\verb|readpkt| method that is used most often because it not only
gets the packet from the underlying socket or file but returns a
proper object to the caller, as opposed to a string of bytes which
much subsequently be passed to the correct packet class's constructor
method.

The implementation of the \verb|PcapConnector| class's
\verb|readkpkt| and \verb|unpack| methods give a clear
demonstration of the power of \verb|PCS| for user programs.  The
code for both functions is shown, combined, in
Figure~\ref{fig:readpkt-and-unpack-code}.

\begin{figure}
  \centering
  \begin{lstlisting}
[readpkt() code]

packet = self.file.next()[1]
return self.unpack(packet, self.dlink, 
                   self.dloff)

[... and now unpack() ...]

if dlink == pcap.DLT_EN10MB:
    return packets.ethernet.ethernet(packet)
elif dlink == pcap.DLT_NULL:
    return packets.localhost.localhost(packet)
else:
    raise UnpackError, 
        "Could not interpret packet"
      \end{lstlisting}
      \caption{readpkt and unpack method code}
      \label{fig:readpkt-and-unpack-code}
\end{figure}

The \verb|readpkt| method actually knows nothing about packets, it
only knows about the underlying interface, a pcap connection to a
network interface or file.  The \verb|unpack| method is what maps a
datalink type, such as Ethernet or Localhost (also known as the
loopback interface) to a packet type and attempts to construct the
proper packet chain for each packet read from the file or the wire.
Successive calls of the \verb|next| method in each object correctly
created will cause an entire chain of packets to be returned to the
caller.  Visualizing the packet chain will be easier when we give a
concrete example in Section~\ref{sec:a-concrete-example}.

Writing, or injecting, packets into a file or the network is handled
more simply because each packet already has within it the proper set
of bytes to write to the file or network interface.  Each connector
class need only implement a \verb|write| method, which takes as its
arguments the set of bytes and the length of the bytes, in order to
properly output the packet.

\section{A Concrete Example}
\label{sec:a-concrete-example}

The best way to show the power of \verb|PCS| is to show it being
used in a real program.  The Packet Debugger (\verb|pdb|) is a
program which works with packet capture files, those read and written
with the \verb|pcap| library, as if they are lines of source code.
With \verb|pdb| a user can load, manipulate, edit and replay
\verb|pcap| files in order to debug their network protocol, or any
other network based software.  The \verb|pdb| program uses all of
the previously described features of \verb|PCS| to achieve these
goals, in less than 1000 lines of code.

\verb|pdb| operates on streams of packets which originate either in
pre-recorded \verb|pcap| files or are captured live from the
network.  Each stream is an array of packets, or packet chains, which
can be loaded, unloaded, inspected, updated and replayed.  Every
attempt has been made to replicate the source code debugger metaphor.
A user can set a break point at a particular packet in the stream
(\verb|break|), replay the stream (\verb|run|), print the stream
(\verb|list|), print the packet (\verb|print|), and send individual
packets (\verb|send|).

In order to load a stream the user has to specify a previously saved
\verb|pcap| file, either from the shell, when the program is
started, or from \verb|pdb's| command line interpreter.  The
\verb|load| command causes the constructor of the \verb|Stream|
object to be called which then loads the stream of packets from the
file as seen in Figure\ref{fig:loading-packets-from-a-file}.

\begin{figure}
  \centering
\begin{lstlisting}
if (filename != None):
    self.file = pcs.PcapConnector(filename)

    done = False
    packet = None
    while not done:
        try:
            packet = self.file.readpkt()
        except:
            done = True

        self.packets.append(packet)

    self.file.close()
  \end{lstlisting}
  \caption{Loading packets from a file}
  \label{fig:loading-packets-from-a-file}
\end{figure}

The code that loads the packets actually does not understand anything
about the packets or packet classes that it is loading, as would be
expected with a clean separation of responsibilities.  In \verb|pdb|
every attempt is made to \emph{not} know things about the internals of
packets and to treat streams of packets as if they are just a
collection of opaque objects.

When the debugger displays packets, either as a listing or as a single
packet, it depends on \verb|PCS| and its packet classes to handle
properly displaying the packet data.  An example listing is shown in
Figure\ref{fig:packet-listing}.

\begin{figure}
  \centering
\begin{lstlisting}
pdb> list
0: <Ethernet: src: '\x00\r\x022{\x9c', 
   dst: '\xff\xff\xff\xff\xff\xff', type: 2054>

  <ARP: spa: 3538819329L, tpa: 3538819566L, 
   hln: 6, pro: 2048, sha: '\x00\r\x022{\x9c', 
   pln: 4, hrd: 1, 
   tha: '\x00\x00\x00\x00\x00\x00', op: 1>

1: <Ethernet: src: '\x00\r\x022{\x9c', 
   dst: '\xff\xff\xff\xff\xff\xff', type: 2054>

  <ARP: spa: 3538819329L, tpa: 3538819566L, 
   hln: 6, pro: 2048, sha: '\x00\r\x022{\x9c', 
   pln: 4, hrd: 1, 
   tha: '\x00\x00\x00\x00\x00\x00', op: 1>
 \end{lstlisting}
 \caption{Packet Listing in pdb}
  \label{fig:packet-listing}
\end{figure}

The code in \verb|pdb| which handles the display of packets does
not know anything about the packet format itself, it calls the
\verb|println| method of the relevant packet object and walks the
packet's \verb|data| chain, as explained in
Section~\ref{sec:unpacking-layers-of-packets} and
Figure~\ref{fig:data-field-chain}, to display all of the relevant
fields.

When \verb|pdb| sends packets it again exploits features of
\verb|PCS| so that it does not need to look into the packets
themselves.  Because packet objects in \verb|PCS| always keep their
\verb|bytes| attributes up to date it is possible to use a single
line of code in order to transmit them, as seen in
Figure~\ref{fig:transmitting-a-packet}.

\begin{figure}
  \centering
\begin{verbatim}
written = self.outfile.write(packet.bytes, len(packet.bytes))
\end{verbatim}
  \caption{Transmitting a packet}
  \label{fig:transmitting-a-packet}
\end{figure}

\section{Related Work}
\label{sec:related-work}

Prolac is a language specifically written for writing network
protocols, and a TCP protocol implementation has been written in it
\cite{kohler:sigcomm99}.  Prolac is a functional programming language
which is related to ML, Haskell and Lisp.  The performance reached by
the Prolac based TCP was comparable to Linux at the time of the paper,
1999, but no subsequent comparison has been made since.

Packet types \cite{mccann:sigcomm00}, is a system that was built to
make working with packets in systems software, written in C.  It
included its own language for laying out packets.  The goal of the
system and its compiler were to generate code that could be used in a
router or operating system kernel.

The Click Modular Router \cite{kohler:tocs00} attempts to make
implementing network forwarding systems, such as routers, simpler. .
The Click system is a set of C++ classes which implement discrete
components of protocol processing, as they would be used by a router,
such as forwarding table lookups, local vs. non-local routing
decisions and interface selection.  Click does not attempt to expose
packet field access in any special or meaningful way to the protocol
writer.

\section{Current Status and Future Work}
\label{sec:future-work}

At the time of this writing \verb|PCS| has been used to write
protocol fuzzers for the IPv6 family of protocols, and implement the
packet debugger described in this paper.  Both \verb|PCS| and
\verb|pdb| are distributed under a BSD license and are hosted on
the Source Forge open source project hosting system at
\url|http://pcs.sf.net| and \url|http://pktdbg.sf.net| respectively.
\verb|PCS| currently supports many well known packet formats,
including Ethernet, IPv4, IPv6, ICMPv4, ICMPv6, Neighbor Discovery,
UDP, TCP and DNS.

Future work with \verb|PCS| will include adding more packet formats
to the base library with the eventual goal of covering 90\% of the
most popular network protocols, including HTTP, DHCP, SCTP, Apple
Talk, IPX/SPX, various instant messaging protocols and others.  One of
the first priorities will be to add proper TCP reassembly so that
higher level protocols, such as HTTP, can be treated as discrete
packets in their own rights.

\verb|PCS| will be used to continue to expand the capabilities of
\verb|pdb|, including the live updating of packets that have been
loaded into the debugger, and will also be the base library for an
upcoming protocol conformance tester.

\section{Conclusions}
\label{sec:conclusions}

In this paper we have presented the Protocol Construction Set (PCS)
which provides a new approach to writing network protocol code and
makes building new protocols, and new protocol tools, easier and less
error prone.  We have shown how PCS's two key features, a simple and
clean representation of packets and packet fields in source code, and
the promotion of packets to high level language objects are used in
implementing a complex networking tool, the packet debugger.  We
believe that our approach to separating representation from
implementation and making packets easier to work with will make it
possible for people to experiment and build protocols and protocol
tools more quickly and more effectively.

\bibliographystyle{abbrv}
\bibliography{gnn}

\end{document}
%%% Local Variables:
%%% mode: latex
%%% TeX-master: t
%%% End:
