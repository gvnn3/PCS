\documentclass[11pt]{article}
\usepackage{codespelunking}
\usepackage[pdftex]{hyperref}
\hypersetup{ 
    pdfauthor   = {George V. Neville-Neil},%
    pdftitle    = {Network Protocol Security Testing with the Packet
      Construction Set},%
    pdfkeywords = {vulnerabilities, attacks, TCP/IP},%
    pdfcreator  = {PDFLaTeX},%
    pdfproducer = {PDFLaTeX}
}
\title{Network Protocol Security Testing with the Packet Construction Set}
\author{George V. Neville-Neil}
\begin{document}
\maketitle
\begin{abstract}
  Writing network protocol code is hard work.  Testing network
  protocol conformance and security is often as hard as writing the
  protocol code itself.  We introduce the Protocol Construction Set
  \program{PCS}, a set of \program{Python} modules which alleviate
  much of the work in writing network protocols and protocol tests.  A
  brief introduction to working with \program{PCS} is given and then
  two example tests are shown as well as a script to analyze data from
  potential distributed denial of service (DDOS) attacks.
\end{abstract}

\section{Introduction}
\label{sec:introduction}

Writing network protocol code is difficult work.  A typical network
protocol, such as TCP/IP, found in FreeBSD \cite{FreeBSD}, contains 50
files with approximately 50,000 lines of C code without including any
device drivers or supporting software such as the routing table.  Most
network protocols are specified in separate documents and the code and
specification do not always match.  How then does one go about
verifying a network protocol, or discovering if it has any security
holes?

For some network protocols, those that have very high market values,
there are specialized, and expensive tools such as \cite{anvl} and
\cite{smartbits}, but these tools are often not available to the
average researcher and do not specifically look for security problems.
In order to rigorously test protocols it becomes necessary to create
a duplicate of the original code and test the first with the second.
Creating a single protocol stack is hard enough, but creating two
seems to be ridiculous and so what most people do is either test the
protocol code by using it or to write specialized tests which only
test known problem areas, neither of these is sufficient.

What is really called for is a programming language or system in which
protocol code, such as packets, are first class objects so that
the tests can written more naturally and with less supporting code.

Writing languages is possibly more difficult that writing network
protocol code, and is rarely undertaken by the same people.  Another
problem with proposing a new language is that about $90\%$ of the
keywords and language constructs for such a \emph{packet language}
would repeat keywords and constructs in pre-existing languages.
Specialized languages also have a lower uptake rate by programmers,
because no one wants to learn the ins and outs of a narrow use
language unless they absolutely have a need to.  For these reasons we
undertook to write a system to bring packet objects into a
pre-existing language, \program{Python}\cite{python} a widely used,
object oriented, interpreted programming language.

\program{Python} was chosen for several reasons.  First it allows the
programmer to create new objects easily and provide those to other
programmers.  Second, it is a powerful and and fully expressive
language that has all the programming language features that modern
programmers expect.  Third, unlike some other scripting languages, it
is easy to read and document, and lastly it has a built in unit test
facility meaning a separate testing framework is unnecessary for most
uses.

The goals of the \program{PCS} project are to provide:

\begin{itemize}
\item Make packets first class objects in the system
\item Provide simple ways to create packet objects
\item Make it easy for programmers to work with packets
\item Cover $80\%$ of the most popular network protocols
\end{itemize}

\newpage
\section{Programming with PCS}
\label{sec:programming-with-pcs}

\begin{figure}[t]
  \centering
\begin{verbatim}
    0                   1                   2                   3   
    0 1 2 3 4 5 6 7 8 9 0 1 2 3 4 5 6 7 8 9 0 1 2 3 4 5 6 7 8 9 0 1 
   +-+-+-+-+-+-+-+-+-+-+-+-+-+-+-+-+-+-+-+-+-+-+-+-+-+-+-+-+-+-+-+-+
   |Version|  IHL  |Type of Service|          Total Length         |
   +-+-+-+-+-+-+-+-+-+-+-+-+-+-+-+-+-+-+-+-+-+-+-+-+-+-+-+-+-+-+-+-+
   |         Identification        |Flags|      Fragment Offset    |
   +-+-+-+-+-+-+-+-+-+-+-+-+-+-+-+-+-+-+-+-+-+-+-+-+-+-+-+-+-+-+-+-+
   |  Time to Live |    Protocol   |         Header Checksum       |
   +-+-+-+-+-+-+-+-+-+-+-+-+-+-+-+-+-+-+-+-+-+-+-+-+-+-+-+-+-+-+-+-+
   |                       Source Address                          |
   +-+-+-+-+-+-+-+-+-+-+-+-+-+-+-+-+-+-+-+-+-+-+-+-+-+-+-+-+-+-+-+-+
   |                    Destination Address                        |
   +-+-+-+-+-+-+-+-+-+-+-+-+-+-+-+-+-+-+-+-+-+-+-+-+-+-+-+-+-+-+-+-+
   |                    Options                    |    Padding    |
   +-+-+-+-+-+-+-+-+-+-+-+-+-+-+-+-+-+-+-+-+-+-+-+-+-+-+-+-+-+-+-+-+
\end{verbatim}
  \caption{IPv4 Header Format}
  \label{fig:rfc791-header}
\end{figure}

One of the major goals of PCS is to make it easy for programmers to
work with packets.  We believe that giving the programmer simple
programmatic access to the fields of the packet is the best way to
make writing protocol code simpler and to make reading it nearly
trivial. Consider the \program{IPv4} packet shown in
\ref{fig:rfc791-header}.  

The IPv4 packet is an excellent example of the problems a programmer
with encounter in writing protocol code because it has several
problems including fields that:

\begin{itemize}
\item are less than one byte in length \emph{Version, Header Length etc.}.
\item do not start on a byte boundary. \emph{Fragment Offset}
\item are an odd number of bits. \emph{Flags}
\end{itemize}


All of which require special macros in languages like C or classes in
C++ which muddy up the meaning of the code.

What would be nice, from the programmers point of view, is to be able
to read and write values from fields in a packet that are approriately
named, have automatic field overflow checking, and be able to then
transmit or receive the packet through a simple interface.
Example~\ref{fig:field-access-and-packet-construction} shows an
example of how \program{PCS} provides this for IPv4 packets.

\begin{figure}
  \centering
\begin{verbatim}
>>> ip = ipv4() # Create a new IPv4 object.
>>> print ip.ttl # Access a field
64
>>> ip.bytes
'@\x00\x00\x00\x00\x00\x00\x00@\x00\x00\x00\x00\x00\x00\x00\x00\x00\x00\x00'
>>> ip.ttl = 32
32
>>> ip.bytes
'@\x00\x00\x00\x00\x00\x00\x00 \x00\x00\x00\x00\x00\x00\x00\x00\x00\x00\x00'
\end{verbatim}
  \caption{Field Access and Packet Construction}
  \label{fig:field-access-and-packet-construction}
\end{figure}

In Figure~\ref{fig:field-access-and-packet-construction} we see a new
IP packet being created from an \class{ipv4} class on line one.  We'll
come to to how packet classes are created in a moment.  Now that we
have an instance of an IPv4 packet we can work with its fields.  On
line 2 of the example we print the value of the \field{ttl} field,
which indicates the packet's \emph{time to live}.  At line 4 we show
the \field{bytes} field of the \object{ip} object.  The \field{bytes}
field always contains the most up to date bits that make up the
packet.  Note the 9th byte in the string, and that it is set to the
character @.  On line 5 we set the \field{ttl} field to 32 and at line
6 we again print the bytes of the packet.  The 9th byte is now a space
because the underlying packet as been changed.  We have now shown how
a programmer would work with a packet but how can the programmer
create packets of their own?

To create a packet the programmer sets up a special class, which
inherits from the \class{Packet} base class.  

\begin{figure}
  \centering
\begin{verbatim}
class ipv4(pcs.Packet):

    layout = pcs.Layout()

    def __init__(self, bytes = None):
        """ define the fields of an IPv4 packet, from RFC 791
        This version does not include options."""
        version = pcs.Field("version", 4, default = 4)
        hlen = pcs.Field("hlen", 4)
        tos = pcs.Field("tos", 8)
        length = pcs.Field("length", 16)
        id = pcs.Field("id", 16)
        flags = pcs.Field("flags", 3)
        offset = pcs.Field("offset", 13)
        ttl = pcs.Field("ttl", 8, default = 64)
        protocol = pcs.Field("protocol", 8)
        checksum = pcs.Field("checksum", 16)
        src = pcs.Field("src", 32)
        dst = pcs.Field("dst", 32)
        pcs.Packet.__init__(self,
                            [version, hlen, tos, length, id, flags, offset,
                             ttl, protocol, checksum, src, dst],
                            bytes = bytes)
        # Description MUST be set after the PCS layer init
        self.description = "IPv4"
\end{verbatim}
  \caption{IPv4 Packet Class \method{\_\_init\_\_} Method}
  \label{fig:ipv4-packet-init-method}
\end{figure}

The heart of each packet class is the set of bit fields which describe
it.  Figure~\ref{fig:ipv4-packet-init-method} shows the
\method{\_\_init\_\_} method for the \class{ipv4} class which
describes the fields in an IPv4 packet.  Each field has a name, a
width in bits, and may have a default value.  Note that bit fields can
be of any size, from 1 to 32, and can be aligned anywhere in the
packet.  The \program{PCS} modules handle all the necessary
conversions between program values and the underlying bits in the
packet.  All that is necessary for the programmer to do is to give the
layout of the packet and \program{PCS} will reproduce it faithfully.

The IPv4 packet is a particularly good example to work with because it
has several odd fields.  Referring back to
Figure~\ref{fig:rfc791-header} we see that the first two fields are less
than a byte in length, each is 4 bits.  The \field{flags} field is 3
bits and the \field{offset} field is 13 bits and does not start on a
byte boundary.  All of these issues mean that in most languages, such
as \program{C} we have to have specialized macros to properly access
these fields, and these macros can decrease the readability of code.
While it is very important to use a high performance language to
implement a protocol, it is not strictly necessary for testing
protocols, and this is one of the trade offs that \program{PCS} has
made.

\section{A Simple Example}
\label{sec:a-simple-example}

As with any new programming system the easiest way to get a grasp on
what it is doing is to give a simple example.
Figure~\ref{fig:sending-a-fake-arp-packet} shows a short program using
\program{PCS} which sends a fake ARP (Address Resolution Protocol)
packet.  ARP is used to resolve IPv4 addresses on a local subnet into
hardware addresses so that an Ethernet device can send the packet to
the correct node.  There are several attacks against ARP but this
example only shows us sending a fake packet.

\begin{figure}
  \centering
\begin{verbatim}
from pcs.packets.arp import *
from pcs.packets.ethernet import *

arppkt = arp()
arppkt.op = 1
arppkt.sha = ether_atob(ether_source)
arppkt.spa = inet_atol(ip_source)
arppkt.tha = "\x00\x00\x00\00\x00\x00"
arppkt.tpa = inet_atol(target)

ether = ethernet()
ether.src = ether_atob(ether_source)
ether.dst = "\xff\xff\xff\xff\xff\xff"
ether.type = 0x806

packet = Chain([ether, arppkt])

output = PcapConnector(interface)

out = output.write(packet.bytes, len(packet.bytes))
\end{verbatim}
  \caption{Sending a Fake ARP Packet}
  \label{fig:sending-a-fake-arp-packet}
\end{figure}

The 20 line program in Figure~\ref{fig:sending-a-fake-arp-packet} is a
relatively complete example, except for retrieving some values from
the user.  At the beginning, lines 1 and 2 we need to import the
proper packets to use.  Although it is possible to import all packets
at once, well written scripts will only import the packets that they
need to increase readability.  The script then creates an ARP and an
Ethernet packet and sets their fields to the appropriate values.  The
ARP packet is a request (\field{op} = 1) which is from a user
specified source hardware and IPv4 address and also has a user
specified IPv4 destination address (lines 6, 7 and 9).  In order to
send our packet somewhere useful we need to encapsulate it within an
Ethernet packet.  On lines 13 through 16 we create an Ethernet packet,
give it a user defined source address, set its destination to the
broadcast address and give it a type of $0x806$ which is the Ethernet
protocol type for an ARP packet.

Encapsulation is handled via the \class{Chain} class in \program{PCS}.
A \class{Chain} is one or more packets connected together.  On line 18
we instantiate a \object{packet} object from the \class{Chain} class,
and we give a list of the packets that make up the chain ([ether,
arppkt]) to the constructor method of the class.  The order of the
elements of the list is important as it represents the order of the
packets in the chain.  If, for example, we had placed the
\object{ether} object after the \object{arpppkt} object we would have
stacked the packets in the wrong order.  The returned packet object
has the bytes of both packets in its own \field{bytes} field which we
use on line 20 to transmit the packet.

In between creating the chain and transmitting the packets we used a
class that had not yet been introduced, a \class{Connector}.
Connectors are classes that implement methods for accessing network
sockets and other transmission media.  Although \program{Python} does
give access to TCP and UDP sockets it does not, by default, come with
interfaces to write raw packets.  \program{PCS} uses a specially
adapted version of Doug Song's \program{pypcap} which gives fairly
complete access to raw packet reading and writing abilities.  In our
example we only use the \class{PcapConnector} to write bytes to the
raw network, as we have to bypass the network stack to transmit out
Ethernet packet, but the \class{Connector} classes actually have much
more functionality, just more than we want to describe here.

\section{Security Testing  with PCS}
\label{sec:security-testing-with-pcs}

How does \program{PCS} relate to security?  Once we are able to send
any packet we wish it is far easier to write a set of tools for
security testing at several different levels.  At the operating system
kernel level it is possible to hunt for externally trigger-able kernel
panics, which can lead to denial of service (DOS) attacks.  At the
network protocol level it might be possible to cause information to
leak from the system, through triggering code paths that are not
normally tested.  Once we arrive at the application level the same set
of concerns apply, though crashing the application is not likely to be
as catastrophic as panic'ing the kernel it still leads to the same
problem, denial of service.  Since \program{PCS} allows the programmer
to specify and run protocols at any layer, through the use of various
\class{Connectors} the possibilities are really only limited by the
ingenuity of the programmer.

One way in which \program{PCS} was used recently to test the security
of a system was by Cl\'ement Lecigne who for his Google Summer of Code
project \cite{lecigne} used a combination of \program{PCS} based
scripts and other tools to attempt to find security issues in the
FreeBSD IPv6 protocol stack.  For the project several different
protocol fuzzers were constructed which would test the protocol both
internally, requiring login access to the target of the attack and
externally, attacking the kernel via the IPv6 protocol.

One attack that was tried, and which was partially successful, was to
send an ICMPv6 packet to a target telling it that the maximum transfer
unit of the system with which it was communicating was smaller than it
really was.  This ``too big'' attack could prevent one node from
communicating with another by forcing the attacked node to update its
routing table with a ridiculously small value, one that is too small
for the IPv6 protocol to transmit.  

\begin{figure}
  \centering
\begin{verbatim}
# Putting on the headers
def toobig(iface, mtu, dstip, dstmac, srcip, srcmac, pkt):
    """send fake icmp TOO BIG packet"""
    # ethernet header
    eth = ethernet()
    eth.dst = eth.name2eth(dstmac)
    eth.src = eth.name2eth(srcmac)
    eth.type = ETHERTYPE_IPV6
    # ipv6 header
    ip = ipv6.ipv6()
    ip.traffic_class = 0
    ip.flow = 0
    ip.next_header = IPPROTO_ICMPV6
    ip.hop = 255
    ip.length = 8 + len(pkt)
    ip.src = inet_pton(AF_INET6, srcip)
    ip.dst = inet_pton(AF_INET6, dstip)

    # icmp6 header
    icmp6 = icmpv6(ICMP6_PACKET_TOO_BIG)
    icmp6.type = ICMP6_PACKET_TOO_BIG 
    icmp6.code = 0
    icmp6.mtu = mtu
    icmp6.checksum = icmp6.cksum(ip, pkt)
    chain = pcs.Chain([eth, ip, icmp6])
    c = pcs.Connector("IPV6", iface)
    c.write(chain.bytes + pkt)
    c.close()
\end{verbatim}
  \caption{IPv6 Packet Attack}
  \label{fig:ipv6-packet-attack}
\end{figure}

Figure~\ref{fig:ipv6-packet-attack} shows a section of code from the
\program{toobig} attack.  The \function{toobig} function takes several
supplied arguments and creates a two packet chain with an IPv6 packet
and an ICMPv6 packet.  The ICMPv6 packet contains the attack in that
it sets the \field{icmp6.type} field to
\constant{ICMP6\_PACKET\_TOO\_BIG} and then sets the \field{icmp6.mtu}
field to whatever the user supplies.

Although this code does not cause a kernel panic it can cause two
nodes to be unable to communicate until they are rebooted.

\subsection{Working with Dump Files}
\label{sec:working-with-dumpfiles}

While \program{PCS} provides a handy way to generate attacks through
the creation of arbitrary packets it can also be used to detect
attacks by reading packets from the network or from a pcap dump file.
Currently most of the tools that work with pcap dump files, most often
created by the \program{tcpdump} program and read by the
\program{ethereal} packet sniffer, are simple command line programs
which do one or two things with the dump file.  With \program{PCS's}
built in pcap support and ability to work pragmatically with
packets we can implement a distributed denial of service detector
program that can read either from the network or a dump file.

Why do we need a program to know if we're under DDOS?  Occasionally
a DDOS will really be a misconfigured server or router and so the
higher network traffic that leads someone to think they're under a
DDOS attack may all be coming from one, or a small number of
addresses.  Since the machines sending the packets are your own, and
are on your network, they are nearby and can generate an inordinately
high load.

In order to verify a DDOS attack all that is required is to analyze
the source address of the packets.  If they are all from one or two
systems, or if the packets come from within one of your own
subnetworks then you are actually DOSing yourself and you need to
fix your equipment.

\begin{figure}
  \centering
\begin{verbatim}
? ddos\_analyze.py -f pcaptestfile -s 255.255.255.0 -n 10.0.0.0 -m 5

5001 packets in dumpfile
5 unique source IPs
0 packets in specified network
Top 5 source addresses were
Address 204.152.184.203 Count 2473 Percentage 49.450110
Address 64.13.135.16    Count 2    Percentage 0.039992
Address 64.13.134.241   Count 1    Percentage 0.019996
Address 195.137.95.246  Count 1    Percentage 0.019996
Address 64.13.134.241   Count 1    Percentage 0.019996
Address 195.137.95.246  Count 1    Percentage 0.019996
1.898u 0.214s 0:02.12 99.0%     0+0k 0+7io 0pf+0w
\end{verbatim}
  \caption{Test run of a DDOS analyzer}
  \label{fig:test-run-of-ddos-analyzer}
\end{figure}

Figure~\ref{fig:test-run-of-ddos-analyzer} shows the output of our
DDOS analyzer being run against a 5000 packet pcap file (snaplen of
9000 bytes) taken from a public server.  We can see that the server is
not under DDOS, but is instead receiving almost $50\%$ of its traffic
from a single IP address.  The analysis took 2 seconds on a 2GHz Mac
Book Pro Core Duo.

\begin{figure}
  \centering
\begin{verbatim}
while not done:
    try:
       packet = file.read()
    except:
       done = True
    packets += 1
    ip = ipv4(packet[file.dloff:len(packet)])
    if (ip.src & mask) != network:
        if ip.src in srcmap:
           srcmap[ip.src] += 1
        else:
           srcmap[ip.src] = 1
    else:
        in_network +=1

# Doing the analysis
hit_list = sorted(srcmap.itervalues(), reverse = True)
for i in range(1,max):
    for addr in srcmap.items():
        if addr[1] == hit_list[i]:
           print "Address %s\t Count %s\t Percentage %f" 
                  % (inet_ntop(AF_INET, struct.pack('!L', addr[0])), 
                     addr[1], 
                    (float(addr[1]) / float(packets)) * float(100))

\end{verbatim}
  \caption{DDOS Analysis Code}
  \label{fig:ddos-analysis-code}
\end{figure}

Figure~\ref{fig:ddos-analysis-code} shows the core of the
\program{ddos\_analyze.py} program.  The program reads all the packets
from the dump file, and then checks to see whether the source address
is within the user's subnet or not (line 8).  A dictionary is used to
store each unique source address found, and a count of the times that
address has been seen (lines 9 through 12).  Lines 17 through 21 do
the actual analysis by first sorting the addresses from most seen to
least and then showing the top N addresses, where N is specified by
the users.  The total size of the full script is 67 lines including
options parsing.

\section{Currents Status}
\label{sec:current-status}

Development on \program{PCS} is ongoing.  At the moment Alpha 0.3
currently available on the \href{http://pcs.sf.net}{PCS} web site
which is hosted by Source Forge.  The list of packets supported is
given in Figure~\ref{fig:packets-supported-by-pcs}.

\begin{figure}
  \centering
  \begin{itemize}
  \item Link Layer: Localhost, Ethernet
  \item Network Layer: ARP, IPv4, ICMPv4, IPv6, ICMPv6, ND6
  \item Transport Layer: UDP, TCP
  \item Application Protocols: DNS, DHCPv4
  \end{itemize}
  \caption{Packets Supported by \program{PCS}}
  \label{fig:packets-supported-by-pcs}
\end{figure}

Every protocol has a test suite supported by the \program{Python}
unittest framework.  Several short scripts are included with the
distribution and these are shown in
Table~\ref{fig:current-pcs-scripts}.

\begin{figure}
  \centering
\begin{tabular}{|l|l|}
\hline
Script & Purpose \\
\hline
arpwhohas.py & Generate a fake ARP\\
ddos\_analyze.py & Determine majority source addresses in a pcap file \\
dns\_query.py & Generate a fake DNS query\\
http\_get.py & Grab a web page\\
pcap\_info.py & Print out various bits of info about a pcap file \\
pcap\_slice.py & Carve up pcap files analogous to tcpslice\\
ping.py & A simple ICMPv4 packet generator \\
snarf.py & A trivial packet sniffer \\
udp\_echo.py & Generate a fake UDP packet\\
\hline
\end{tabular}
  \caption{Current \program{PCS} Scripts}
  \label{fig:current-pcs-scripts}
\end{figure}

\section{Future Work}
\label{sec:future-work}

The focus on the near future is to add more packet classes to the
system and attempt to cover $80\%$ of all known protocols.  Secondary
to packet coverage we intend to add more scripts but what we really
want is for more people to incorporate \program{PCS} into their own
code rather than providing a complete set of scripts with the
library.  First and foremost \program{PCS} is a programming system and
we want programmers to use it.

Although there is documentation, tests and scripts to read the
documentation can always be improved and this will continue apace with
the code itself.

\section{Related Work}
\label{sec:related-work}

Several projects attempt to build packet manipulation libraries for
use by programmers.  The \program{Libnet} \cite{libnet} and
\program{Libnet-ng} libraries are written in \program{C} and give the
programmer the ability to extend them by adding more \program{C}
functions to the library.  \program{dpkt} \cite{dpkt} and
\program{dnet} \cite{libdnet} are both Python libraries that work in much
the same way as \program{Libnet} does, in that the provide access to
packet construction APIs.  \program{scapy} \cite{scapy} is a Python
library that gives a large system for building packets who's goal is
fuzzing and security checking.  All of these systems provide only
partial solutions to the problem that \program{PCS} is trying to solve
and none of them make creating packet classes any easier.

\section{Availability}
\label{sec:availability}

The PCS project is hosted on Source Forge and can be found at
http://pcs.sf.net.  It is distributed under the BSD License.

\begin{thebibliography}{99} 
\bibitem{rfc791} Postel J.: RFC 791 Internet Protocol, September 1981
\bibitem{FreeBSD} McKusick K., Neville-Neil G.: The Design and
  Implementation of the FreeBSD Operating System, Addison-Wesley
  Longman, August 2004
\bibitem{python} Van Rossum G.: The Python Language Reference Manual,
  Network Theory Ltd. 2003
\bibitem{lecigne} Lecigne C., Neville-Neil G.: Walking through the
  FreeBSD IPv6 Stack (Currently Unpublished)
\bibitem{anvl} ANVL Protocol Tester:
  http://www.phoenixdatacom.com/anvl/anvl.html
\bibitem{smartbits}Spirent Smartbits http://www.spirentcom.com/
\bibitem{scapy} Scapy Web Page: http://www.secdev.org/projects/scapy/
\bibitem{libnet} Libnet Web Page: http://libnet.sourceforge.net/
\bibitem{libdnet} Libdnet Web Page: http://libdnet.sourceforge.net/dnet.html
\bibitem{dpkt} dpkt web page: http://www.monkey.org/~dugsong/dpkt
\end{thebibliography}

\end{document}
