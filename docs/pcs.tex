\documentclass[11pt]{article}
\usepackage{fancyvrb}
\usepackage{listings}
\usepackage{codespelunking}
\usepackage[pdftex]{hyperref}
\title{Packet Construction Set}
\author{George V. Neville-Neil}
\begin{document}
\maketitle
\tableofcontents
\begin{abstract}
  We had Ethernet headers, IP packets, TCP segments, a gaggle of HTTP
  requests and responses, also UDP, NTP, and DHCP. Not that we needed
  all that just to communicate but once you get locked into a serious
  packet collection the tendency is to push it as far as you can. -
  Deepest apologies to Hunter S. Thompson
\end{abstract}
% All code in this file is in Python, set the listings package correctly
\lstset{language=Python, escapeinside={(*@}{@*)}, numbers=left}
\section{Introduction}

PCS is a set of Python modules and objects that make building network
protocol testing tools easier for the protocol developer.  The core of
the system is the pcs module itself which provides the necessary
functionality to create classes that implement packets.

Installing PCS is covered in the text file, \file+INSTALLATION+, which came
with this package.  The code is under a BSD License and can be found
in the file \file+COPYRIGHT+ in the root of this package.

In the following document we set \class+classes+ 
\function+functions+ and \method+methods+ apart by setting them in
different type.  Methods and functions are also followed by
parentheses, ``()'', which classes are not.


\section{A Quick Tour}

For the impatient programmer this section is a 5 minute intro to using
PCS.  Even faster than this tour would be to read some of the test
code in the \file+tests+ sub-directory or the scripts in the
\file+scripts+ sub directory.

PCS is a set of functions to encode and decode network packets from
various formats as well as a set of \emph{classes} for the most
commonly use network protocols.  Each object derived from a packet has
fields automatically built into it that represent the relevant
sections of the packet.  

Let's grab a familiar packet to work with, the IPv4 packet.  IPv4
packets show a few interesting features of PCS.  Figure
\ref{fig:rfc791-ipheader} shows the definition of an IPv4 packet
header from \cite{rfc791} which specifies the IPv4 protocol.  

\begin{figure}
\label{fig:rfc791-ipheader}
  \centering
\begin{Verbatim}
    0                   1                   2                   3   
    0 1 2 3 4 5 6 7 8 9 0 1 2 3 4 5 6 7 8 9 0 1 2 3 4 5 6 7 8 9 0 1 
   +-+-+-+-+-+-+-+-+-+-+-+-+-+-+-+-+-+-+-+-+-+-+-+-+-+-+-+-+-+-+-+-+
   |Version|  IHL  |Type of Service|          Total Length         |
   +-+-+-+-+-+-+-+-+-+-+-+-+-+-+-+-+-+-+-+-+-+-+-+-+-+-+-+-+-+-+-+-+
   |         Identification        |Flags|      Fragment Offset    |
   +-+-+-+-+-+-+-+-+-+-+-+-+-+-+-+-+-+-+-+-+-+-+-+-+-+-+-+-+-+-+-+-+
   |  Time to Live |    Protocol   |         Header Checksum       |
   +-+-+-+-+-+-+-+-+-+-+-+-+-+-+-+-+-+-+-+-+-+-+-+-+-+-+-+-+-+-+-+-+
   |                       Source Address                          |
   +-+-+-+-+-+-+-+-+-+-+-+-+-+-+-+-+-+-+-+-+-+-+-+-+-+-+-+-+-+-+-+-+
   |                    Destination Address                        |
   +-+-+-+-+-+-+-+-+-+-+-+-+-+-+-+-+-+-+-+-+-+-+-+-+-+-+-+-+-+-+-+-+
   |                    Options                    |    Padding    |
   +-+-+-+-+-+-+-+-+-+-+-+-+-+-+-+-+-+-+-+-+-+-+-+-+-+-+-+-+-+-+-+-+
\end{Verbatim}
  \caption{IPv4 Header Format}
\end{figure}

In PCS every packet class contains fields which represent the fields
of the packet exactly, including their bit widths.
Figure\ref{fig:ipv4-quick-and-dirty} shows a command line interaction
with an IPv4 packet.

\begin{figure}
  \centering
\begin{lstlisting}
>>> from pcs.packets.ipv4 import *
>>> ip = ipv4()
>>> print ip
version 4
hlen 0
tos 0
length 0
id 0
flags 0
offset 0
ttl 64
protocol 0
checksum 0
src 0.0.0.0
dst 0.0.0.0

>>> ip.hlen=5<<2
>>> print ip
version 4
hlen 20
tos 0
length 0
id 0
flags 0
offset 0
ttl 64
protocol 0
checksum 0
src 0.0.0.0
dst 0.0.0.0
\end{lstlisting}
  \caption{Quick and Dirty IPv4 Example}
  \label{fig:ipv4-quick-and-dirty}
\end{figure}

Each packet has a built in field called \field+bytes+ which always contains
the wire representation of the packet.  

\begin{figure}
  \centering
\begin{Verbatim}
>>> from pcs.packets.ipv4 import *
>>> ip = ipv4()
>>> ip.bytes
'@\x00\x00\x00\x00\x00\x00\x00@\x00\x00\x00\x00\x00\x00\x00\x00\x00\x00\x00'
>>> ip.hlen = 5 << 2
>>> ip.bytes
'D\x00\x00\x00\x00\x00\x00\x00@\x00\x00\x00\x00\x00\x00\x00\x00\x00\x00\x00'
\end{Verbatim}
  \caption{The bytes Field of the Packet}
  \label{fig:bytes-field}
\end{figure}

In Figure\ref{fig:bytes-field} the \field+bytes+ field has been
changed in its first position by setting the \field+hlen+ or header
length field to 20, $5 \ll 2$.  Such programmatic access is available
to all fields of the packet.

The IPv4 header has fields that can be problematic to work with in any
language including ones that are

\begin{list}{fig:ipheadfeatures}{}
\item less than one byte (octect) in length (Version, IHL, Flags)
\item not an even number of bits (Flags)
\item not aligned on a byte boundary (Fragment Offset)
\end{list}

Using just these features it is possible to write complex programs in
Python that directly manipulate packets.  For now you should know
enough to safely ignore this documentation until you to explore further.

\section{Working with Packets}

In PCS every packet is a class and the layout of the packet is defined
by a Layout class which contains a set of Fields.  Fields can be from
1 to many bits, so it is possible to build packets with arbitrary
width bit fields.  Fields know about the widths and will throw
exceptions when they are overloaded.

Every Packet object, that is an object instantiated from a specific
PCS packet class, has a field named bytes which shows the
representation of the data in the packet at that point in time.  It is
the bytes field that is used when transmitting the packet on the wire.

The whole point of writing PCS was to make it easier to experiment
with various packet types.  In PCS there are packet classes and packet
objects.  Packet classes define the named fields of the packet and
these named fields are properties of the object.  A practical example
may help.  Given an IPv6 packet class it is possible to create the
object, set various fields, as well as transmit and receive the
object.

A good example is the IPv6 class:
\begin{figure}
  \centering
\begin{lstlisting}
ip = ipv6()
assert (ip != None)
ip.traffic_class = 1
ip.flow = 0
ip.length = 64
ip.next_header = 6
ip.hop = 64
ip.src = inet_pton(AF_INET6, "::1")
ip.dst = inet_pton(AF_INET6, "::1")
\end{lstlisting}
  \caption{IPv6 Example}
  \label{fig:ipv6-example}
\end{figure}
The code in Figure \ref{fig:ipv6-example} gets a new IPv6 object from
the ipv6() class, which was imported earlier, and sets various fields
in the packet.  Showing the bytes field, Figure
\ref{fig:bytes-ipv6-packet} gives us an idea of how well this is
working.

\begin{figure}
  \centering
\begin{Verbatim}
>>> ip.bytes
'`\x10\x00\x00\x00@\x06@\x00\x00\x00\x00\x00\x00\x00\x00\x00\x00\x00\x00\x00
\x00\x00\x01\x00\x00\x00\x00\x00\x00\x00\x00\x00\x00\x00\x00\x00\x00\x00\x01'
\end{Verbatim}
  \caption{Bytes of the IPv6 Packet}
  \label{fig:bytes-ipv6-packet}
\end{figure}

Note that various bits are set throughout the bytes.  The data in the
packet can be pretty printed using the \function+print+ function as
seen in Figure \ref{fig:printing-a-packet} or it can be dumped as a
string directly as seen in Figure\ref{fig:repr-method}.

\begin{figure}
  \centering
\begin{lstlisting}
>>> print ip
version 6
traffic_class 1
flow 0
length 64
next_header 6
hop 64
src ::1
dst ::1
\end{lstlisting}
  \caption{Printing a Packet}
  \label{fig:printing-a-packet}
\end{figure}

\begin{figure}[h]
  \centering
  \begin{Verbatim}
>>> ip
<IPv6: src: 0, dst: 0, traffic_class: 0, flow: 0, length: 0, \
version:6, hop: 0, next_header: 0>
\end{Verbatim}
    \caption{Using the \_\_repr\_\_ method}
    \label{fig:repr-method}
  \end{figure}

\section{Creating Packet Classes}

For a packet to be a part of PCS it must sub-classed from the \class+Packet+
class as seen in Figure \ref{fig:ipv6-class-definition}.  Thoughout
this section we will use the example of a network layer packet, IPv6,
and a packet about the transport layer, DNS.  Using both low and high
level packets should give the reader a good feel for how to add most
of the packets they would be expected to work with.

\begin{figure}
  \centering
\lstinputlisting [firstline=56,lastline=88]{/Volumes/exported/hg/PCS/pcs/packets/ipv6.py}
  \caption{IPv6 Packet Class from pcs/packets/ipv6.py}
  \label{fig:ipv6-class-definition}
\end{figure}

The code in Figure \ref{fig:ipv6-class-definiition} defines a new
class, one that will describe an IPv6 packet, sub-classed from the
\class+Packet+ base class.  There are a small number of reserved field
names that you \emph{must not} use when defining your packets.  For
reference all of the reserved field names are given in
Table~\ref{fig:reserved_fields_and_methods} and most of them will also
be discussed in this section.  Resereved names that are part of a
class, as opposed to an object, are preceeded by an underscore, \_, to
further set them apart.  Do not use underscores to start your
fieldnames. \emph{You have been warned!}

\begin{figure}
  \centering
\begin{tabular}{|l|l|}
\hline
Field Name & Use \\
\hline
\texttt{\_layout} & Used to store the layout of the packet\\
\hline
\texttt{\_map} & Used to demultiplex higher layer packets\\
\hline
\texttt{next} & A method used to unencapsulate higher layer packets\\
\hline
\texttt{bytes} & Storage for the raw bytes of the packet\\
\hline
\texttt{data} & Pointer to the next higher layer packet\\
\hline
\texttt{description} & Textual description of the packet\\
\hline
\end{tabular}
  \caption{Reserved Fields and Methods in PCS}
  \label{fig:reserved_fields_and_methods}
\end{figure}

Each packet class in PCS is defined in a similar way.  After
sub-classing from the \class+Packet+ class, there should be a Python
style text string describing the class.  The fields are defined
next,as shown on lines \ref{list:field_begin} through
\ref{list:field_end}, in the order in which they are stored in the
packet.  Various types of fields are supported by PCS and they are all
covered in Section~\ref{sec:}.  After all of the fields have been
listed, the \class+Packet+ class's \method+init+ method is called with
three arguments.  The \class+self+ object, an array of the fields, in
the order in which they will appear in a packet, and the \field+bytes+
variable that was passed to the packet object's \method+init+ method.
Once the packet is initalized we set its description, on line 20.
The description comes from the Python docstring which is defined at
the beginning of the \method+__init__+ method, in this case, ``IPv6
Packet from RFC2460''.  Although the docstring may be left blank this
is not advised.

Any packet that may contain data at a higher layer, such as a network
packet will then use its \method+next+ method to unencapsulate any
higher layer packets.  On lines \ref{list:next_begin} through
\ref{list:next_end} the \method+init+ method attempts to unencapsulate
any data after the header itself.  Every packet object either has a
valid \field+data+ or it is set to \constant+None+.  Higher level
programs using PCS will check for \field+data+ being set to
\variable+None+ in order to know when they have reached the end of a
packet so it must be set correctly by each packet class.  The
\method+next+ method used here is from the \class+Packet+ base class
but it can also be overridden by a programmer, and this is done in the
\class+TCP+ class which can be found in \file+pcs/packets/tcp.py+.

\subsection{Working with Different Types of Fields}
\label{sec:working_with_different_types_of_fields}

Part of packet initialization is to set up the fields that the packet
will contain.  Fields in PCS are objects in themselves and they are
initialized in different ways, depending on their type.  A brief list
of the currently supported fields is given in
Table~\ref{fig:fields_supported_by_pcs}.

\begin{figure}
  \centering
  \begin{tabular}{|l|l|l|}
    \hline
    Name & Use & Initialiazation Arguments\\
    \hline
    Field & Abitrary Bit Field & Name, Width in Bits, Default Value\\
    \hline
    StringField & String of Bytes & Name, Width In Bits, Default Value\\
    \hline
    LengthValueField & A set of Values with Associated Lengths & Name,
    Width in Bytes, Default Value\\
    \hline
  \end{tabular}
  \caption{Fields Supported by PCS}
  \label{fig:fields_supported_by_pcs}
\end{figure}

Each field has several possible arguments, but the two that are
required are a name, which is the string field specified as the first
argument and a width, which is the second argument.  Note that some
field widths are specified in \emph{bits} and some in \emph{bytes} or

The fields are set by passing them as an array to the PCS base class
initialization method.

It would have been convenient if all network protocol packets were
simply lists of fixed length fields, but that is not the case.  PCS
defines two extra field classes, the \class+StringField+ and the
\class+LengthValueField+.

The \class+StringField+ is simply a name and a width in bits of the
string.  The data is interpreted as a list of bytes, but without an
encoded field size.  Like a \class+Field+ the \class+StringField+ has
a constant size.

Numerous upper layer protocols, i.e. those above UDP and TCP, use
length-value fields to encode their data, usually strings.  In a
length-value field the number of bytes being communicated is given as
the first byte, word, or longword and then the data comes directly
after the size.  For example, DNS~\cite{rfc1035} encodes the domain
names to be looked up as a series of length-value fields such that the
domain name pcs.sourceforge.net gets encoded as 3pcs11sourceforge3net
when it is transmitted in the packet.

The \class+LengthValueField+ class is used to encode length-value
fields.  A \class+LenghtValueField+ has three attributes, its name,
the width in bits of the length part, and a possible default value.
Currently only 8, 16, and 32 bit fields are supported for the length.
The length part need never been set by the programmer, it is
automatically set when a string is assigned to the field as shown
in~\ref{fig:using-a-length-value-field}.

\begin{figure}
  \centering
\lstinputlisting [firstline=102,lastline=114]{/Volumes/exported/hg/PCS/pcs/packets/dns.py}
  \caption{Using a LengthValueField in pcs/packets/dns.py}
  \label{fig:using-a-length-value-field}
\end{figure}

Figure~\ref{fig:using-a-length-value-field} shows both the definition
and use of a \class+LengthValueField+.  The definition follows the
same system as all the other fields, with the name and the size given
in the initialization.  The \class+dnslabel+ class has only one field,
that is the name, and it's length is given by an 8 bit field, meaning
the string sent can have a maximum length of 255 bytes.  

When using the class, as mentioned, the size is not explicitly set.
One last thing to note is that in order to have a 0 byte terminator
the programmer assigns the empty string to a label.  Using the empty
string means that the length-value field in the packet has a 0 for the
length which acts as a terminator for the list.  For a complete
example please review \file+dns\_query.py+ in the \file+scripts+
directory.

\subsection{Built in Bounds Checking}
\label{sec:built-in-bounds-checking}

One of the nicer features of PCS is built in bounds checking.  Once
the programmer has specified the size of the field, the system checks
on any attempt to set that field to make sure that the value is within
the proper bounds.  For example, in Figure \ref{fig:bounds-checking-1}
an attempt to set the value of the IP packet's header length field to
$16$ fails because the header length field is only 4 bits wide and so
must contain a value between zero and fifteen.

\begin{figure}
  \centering
\begin{lstlisting}
>>> from pcs.packets.ipv4 import *
>>> ip = ipv4()
>>> ip.hlen = 16
Traceback (most recent call last):
[...]
pcs.FieldBoundsError: 'Value must be between 0 and 15'
>>> ip.hlen = -1
Traceback (most recent call last):
[...]
pcs.FieldBoundsError: 'Value must be between 0 and 15'
>>> 
\end{lstlisting}
  \caption{Bounds Checking}
  \label{fig:bounds-checking-1}
\end{figure}

\program+PCS+ does all the work for the programmer once they have set
the layout of their packet.

\subsection{Decapsulating Packets}
\label{sec:decapsulating_packets}

One of the key concepts in networking is that of encapsulation, for
example an IP packet can be encapsulated in an Ethernet frame.  In
order to provide a simple way for programmers to specifying the
mapping between different layers of protocols \program+PCS+ provides a
\method+next+ method as part of the \class+Packet+ base class. There
are a few pre-requisites that the programmer must fulfill in order for
the \method+next+ method to do its job.  The first is that at least
one \class+Field+ must be marked as a \emph{discriminator}.  The
descriminator field is the one that the \method+next+ method will use
to decapsulate the next higher layer packet.  The other pre-requisite
is that the programmer define a mapping of the discriminator values to
other packets.  An example seems the best way to make sense of all
this.

\begin{figure}
\lstinputlisting [firstline=56,lastline=79]{/Volumes/exported/hg/PCS/pcs/packets/ethernet.py}
  \caption{The Ethernet Packet Class from pcs/packets/ethernet.py}
  \label{fig:ethernet-packet-class}
\end{figure}

\begin{figure}
  \lstinputlisting [firstline=41,lastline=55]{/Volumes/exported/hg/PCS/pcs/packets/ethernet_map.py}
  \caption{Ethernet Mapping Class from pcs/packets/ethernet\_map.py}
  \label{fig:ethernet-mapping-class}
\end{figure}

Figure \ref{fig:ethernet_packet_and_mapping_classes} shows an
abbreviated and combined listing of the \class+Ethernet+ class and its
associated mapping class.  The full implementation can be found in the 
source tree in the files \fullpath+pcs/packets/ethernet.py+ and
\fullpath+pcs/packets/ethernet_map.py+ respectively.  On line
\ref{list:map_class_variable} a class variable, one that will be
shared across all instances of this object, is created and set to the
map that is defined in the \class+ethernet_map+ module.

The actual mapping of discriminators to higher layer packets is done
in the mapping module.  Line \ref{list:importing_higher_layer_packets}
shows the mapping module importing the higher layer objects, in this
case the \class+ipv4+, \class+ipv6+, and \class+arp+ packets which can
be encapsulated in an Ethernet frame.  The map is really a
\program+Python+ dictionary where the key is the value that the
\method+next+ method expects to find in the field marked as a
\emph{discriminator} in the \class+ethernet+ packet class.  The values
in the dictionary are packet class constructors which will be called
from PCS's \class+Packet+ base class.

If the preceeding discussion seems complicated it can be summed up in
the following way.  A packet class creator marks a single
\class+Field+ as a \emph{discriminator} and then creates a mapping
module which contains a dictionary that maps a value that can appear
as a discriminator to a consctructor for a higher layer packet class.
In the case of Ethernet the discriminator is the \field+type+ field
which contains the protocol type.  An Ethernet frame which contains an
IPv4 packet will have a \field+type+ field containing the value $2048$
in decimal, $0x800$ hexadecimal.  The \class+Packet+ base class in
this case will handle decapsulation of the higher layer packet.

Mapping classes exist now for most packets, although some packets,
such as TCP and UDP, require special handling.  Refer to the
\method+next+ method implementations in \fullpath+pcs/packets/tcp.py+
and \fullpath+pcs/packets/udp.py+ for more information.

\section{Retrieving Packets}
\label{sec:retrieving-packets}

One of the uses of \program+PCS+ is to analyze packets that have
previously stored, for example by a program such as
\program+tcpdump(1)+.  \program+PCS+ supports reading and writing
\program+tcpdump(1)+ files though the
\href+http://monkey.org/~dugsong/pypcap/++pcap+ library written by
Doug Song.  The python API exactly mirrors the C API in that packets
are processed via a callback to a \function+dispatch+ routine, usually
in a loop.  Complete documentation on the \program+pcap+ library can
be found with its source code or on its web page.  This document only
explains \program+pcap+ as it relates to how we use it in
\program+PCS+.

When presented with a possibly unknown data file how can you start?
If you don't know the bottom layer protocol stored in the file, such
as \emph{Ethernet}, \emph{FDDI}, or raw \emph{IP} packets such as
might be capture on a loopback interface, it's going to be very hard
to get your program to read the packets correctly.  The \program+pcap+
library handles this neatly for us.  When opening a saved file it is
possible to ask the file what kind of data it contains, through the
\method+datalink+ method.

\begin{figure}
  \centering
\begin{lstlisting}
>>> import pcap
>>> efile = pcap.pcap("etherping.out")
>>> efile.datalink()
1
>>> efile.datalink() == pcap.DLT_EN10MB
True
>>> lfile = pcap.pcap("loopping.out")
>>> lfile.datalink()
0
>>> lfile.datalink() == pcap.DLT_NULL
True
>>> lfile.datalink() == pcap.DLT_EN10MB
False
>>> 
\end{lstlisting}
  \caption{Determining the Bottom Layer}
  \label{fig:determining-the-bottom-layer}
\end{figure}

In Figure\ref{fig:determining-the-bottom-layer} we see two different
save files being opened.  The first, \file+etherping.out+ is a tcpdump
file that contains data collected on an Ethernet interface, type
\constant+DLT\_EN10+ and the second, \file+loopping.out+ was collected
from the \emph{loopback} interface and so contains no Layer 2 packet
information.  

Not only do we need to know the type of the lowest layer packets but
we also need to know the next layer's offset so that we can find the
end of the datalink packet and the beginning of the network packet.
The \field+dloff+ field of the \class+pcap+ class gives the data link
offset.  Figure\ref{fig:finding-the-datalink-offset} continues the
example shown in Figure\ref{fig:determining-the-bottom-layer} and
shows that the Ethernet file has a datalink offset of 14 bytes, and
the loopback file 4. 

  \begin{figure}
    \centering
\begin{lstlisting}
>>> efile.dloff
14
>>> lfile.dloff
4
>>> 
\end{lstlisting}
    \caption{Finding the Datalink Offset}
    \label{fig:finding-the-datalink-offset}
  \end{figure}

It is in the loopback case that the number is most important.  Most
network programmers remember that Ethernet headers are 14 bytes in
length, but the 4 byte offset for loopback may seem confusing, and if
forgotten any programs run on data collected on a loopback interface
will appear as garbage.

With all this background we can now read a packet and examine it.
Figure \ref{fig:reading-in-a-packet} shows what happens when we create
a packet from a data file.  

\begin{figure}
  \centering
\begin{lstlisting}
>>> ip = ipv4(packet[efile.dloff:len(packet)])
>>> print ip
version 4
hlen 5
tos 0
length 84
id 34963
flags 0
offset 0
ttl 64
protocol 1
checksum 58688
src 192.168.101.166
dst 169.229.60.161
\end{lstlisting}
  \caption{Reading in a Packet}
  \label{fig:reading-in-a-packet}
\end{figure}

In this example we pre-suppose that the packet is an IPv4 packet but
that is not actually necessary.  We can start from the lowest layer,
which in this case is Ethernet, because the capture file knows the
link layer of the data.  Packets are fully decoded as much as possible
when they are read.  

\begin{figure}
  \centering
\begin{lstlisting}
>>> from pcs.packets.ethernet import ethernet
>>> ethernet = ethernet(packet[0:len(packet)])
>>> ethernet.data
<Packet: hlen: 5, protocol: 1, src: 3232261542L, tos: 0, dst: 2850372769L, ttl: 64, length: 84, version: 4, flags: 0, offset: 0, checksum: 58688, id: 34963>
>>> ip = ethernet.data
>>> print ethernet
src: 0:10:db:3a:3a:77
dst: 0:d:93:44:fa:62
type: 0x800
>>> print ip
version 4
hlen 5
tos 0
length 84
id 34963
flags 0
offset 0
ttl 64
protocol 1
checksum 58688
src 192.168.101.166
dst 169.229.60.161
\end{lstlisting}
  \caption{Packet Decapsulation on Read}
  \label{fig:packet-decapsulation-on-read}
\end{figure}

PCS is able to do this via a special method, called \method+next+ and
a field called \field+data+.  Every PCS class has a \method+next+
method which attempts to figure out the next higher layer protocol if
there is any data in a packet beyond the header.  If the packet's data
can be understand and a higher layer packet class is found the
\method+next+ creates a packet object of the appropriate type and
sets the \field+data+ field to point to the packet.  This process is
recursive, going up the protocol layers until all remaining packet
data or higher layers are exhausted.  In
Figure\ref{fig:packet-decapsulation-on-read} we see an example of an
Ethernet packet which contains an IPv4 packet which contains an ICMPv4
packet all connected via their respective \field+data+ fields.

\section{Storing Packets}

This section intentionally left blank.

Need to update \program+pcap+ module to include support for true dump
files.

\section{Sending Packets}
\label{sec:sending-packets}

In \program+PCS+ packets are received and transmitted (see
\ref{sec:sending-packets} using \class+Connectors+.  A
\class+Connector+ is an abstraction that can contain a traditional
network \emph{socket}, or a file descriptor which points to a protocol
filter such as \emph{BPF}.  For completely arbitrary reasons we will
discuss packet transmission first.

In order to send a packet we must first have a connector of some type
on which to send it.  A trivial example is the \file+http\_get.py+
script which uses a \class+TCP4Connector+ to contact a web server,
execute a simple \em{GET} command, and print the results.

\begin{figure}
  \centering
\lstinputlisting [firstline=37,lastline=59]{/Volumes/exported/hg/PCS/scripts/http_get.py}
  \caption{HTTP Get Script from scripts/http\_get.py}
  \label{fig:http-get-script}
\end{figure}

Although everything that is done in the \program+http\_get+ script
could be done far better with \program+Python's+ native HTTP classes
the script does show how easy it is to set up a connector.

For the purposes of protocol development and testing it is more
interesting to look at the \class+PcapConnector+ class, which is used
to read and write raw packets to the network.  Figure
\ref{fig:transmitting-a-raw-ping-packet} shows a section of the
\program+icmpv4test+ test script which transmits an ICMPv4 echo, aka
ping, packet.  

\footnote{Note that on most operating system you need root privileges
  in use the PcapConnector class.}

\begin{figure}
\lstinputlisting [firstline=120,lastline=182]{/Volumes/exported/hg/PCS/tests/icmpv4test.py}
  \caption{Transmitting a Raw Ping Packet from tests/icmpv4test.py}
  \label{fig:transmitting-a-raw-ping-packet}
\end{figure}

The \function+test\_icmpv4\_ping+ function contains a good deal of code
but we are only concerned with the last two lines at the moment.  The
next to the last line opens a raw pcap socket on the localhost,
\em{lo0}, interface which allows us to write packets directly to that
interface.  The last line writes a packet to the interface.  We will
come back to this example again in section \ref{sec:chains}.

\section{Receiving Packets}
\label{sec:receiving-packets}

In order to receive packets we again use the \class+Connector+
classes.  Figure \ref{fig:packet-snarfing-program} shows the simplest
possible packet sniffer program that you may ever see.

\begin{figure}
\lstinputlisting [firstline=37,lastline=59]{/Volumes/exported/hg/PCS/scripts/snarf.py}
  \caption{Packet Snarfing Program}
  \label{fig:packet-snarfing-program}
\end{figure}

The \program+snarf.py+ reads from a selected network interface, which
in this case must be an Ethernet interface, and prints out all the
Ethernet packets and \emph{any upper level packets that PCS knows
  about.}  It is this second point that should be emphasized.  Any
packet implemented in \program+PCS+ which has an upper layer protocol
can, and should, implement a \method+next+ method which correctly
fills in the packet's \field+data+ field with the upper level
protocol.  In this case the upper layer protocols are likely to be
either ARP, IPv4 or IPv6, but there are others that are possible.

\section{Chains}
\label{sec:chains}

We first saw a the \class+Chain+ class in Figure
\ref{fig:transmitting-a-raw-ping-packet} and we'll continue to refer
to that figure here.  \class+Chains+ are used to connect several
packets together, which allows use to put any packet on top of any
other.  Want to transmit an Ethernet packet on top of ICMPv4?  No
problem, just put the Ethernet packet after the ICMPv4 packet in the
chain.  Apart from creating arbitrary layering, \class+Chains+ allow
you to put together better known set of packets.  In order to create a
valid ICMPv4 echo packet we need to have a IPv4 packet as well as the
proper framing for the localhost interface.  When using \program+pcap+
directly even the localhost interface has some necessary framing to
indicate what type of packet is being transmitted over it.  

The packet we're to transmit is set up as a \class+Chain+ that
contains four other packets: localhost, IPv4, ICMPv4, and Echo.  Once
the chain is created it need not be static, as in this example, as
changes to any of the packets it contains will be reflected in the
chain.  In order to update the actual bytes the caller has to remember
to invoke the \method+encode+ method after any changes to the packets
the chain contains.  \footnote{This may be fixed in a future version
  to make \class+Chains+ more automatic.}

\class+Chains+ can also calculate RFC 792 style checksums, such as
those used for ICMPv4 messages.  The checksum feature was used in
Figure~\ref{fig:transmitting-a-raw-ping-packet}.  Because it is common
to have to calculate checksums over packets it made sense to put this
functionality into the \class+Chain+ class.

\section{Displaying Packets}
\label{sec:displaying-packets}

\begin{Verbatim}
To be done, to be done...
\end{Verbatim}

\end{document}
\begin{thebibliography}{99} 
\bibitem{rfc791} Postel J.: 
RFC 791 Internet Protocol
\bibitem{rfc1035} Mockapetris P..: 
RFC 1035 Domain Names - Implementation and Specification
\end{thebibliography}
